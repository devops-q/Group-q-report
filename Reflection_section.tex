\section{Reflection perspective} \label{rp}
\subsection{Software Evolution and Refactoring} %% msro
%% There should be a couple of issues to choose from in the log.
\subsection{Operation} %% msro
%% Discovering problems with usage from mining bot.
Throughout the project, we learned a major lesson about software. Namely, the key to success is automation and fast feedback loops. If we automate testing, integration, and deployment, then the friction to get from a finished software update to a deployed, working service is minimal. This encourages software evolution in small steps, which in turns makes it much easier to spot bugs and find their cause, and bugs are also found a lot earlier.

We encountered some major issues relating to operations:
\begin{itemize}
    \item After implementing volumes, we had an issue where the \texttt{latest id}
    entry used by the API was still in a text file in the code, and thus would
    not be persisted across deployments. We fixed this issue by incorporating
    \texttt{latest id} in the volume as a single-row single-column database
    table. Related issue: \url{https://github.com/devops-q/devops/issues/250}
    \item We found that we didn’t set up authentication for front of loki
    and alloy. We tried to set it up, but without success. We then realized 
    that loki and alloy are part of the same network and don’t need to expose ports,
    so we stopped port exposition for them, preventing unauthorized access.
\end{itemize}

\subsection{Maintenance} %% alpl
%% Staying up to date with issues, and registering problems using logging


%% THROUGHOUT THIS SECTION, WE NEED TO LINK BACK TO OLD ISSUES.


\subsection{Our approach to DevOps} %% phbl 
Throughout this project, we have sought to implement several DevOps practices to increase the efficiency of our workflow. Here our initial goal was to reduce the amount of time between committing code, code being review and subsequently merged, which we found to be a tedious process. One of the core ideals of DevOps is being able to reduce this lead time between committing and merging, such that developers can work faster and in smaller batches while still maintaining high code quality \parencite{handbook}. We achieved this principle through the implementation of continuous integration, providing a \textit{standardisation} measure for our coding quality using linters and tests as a requirement for merging a commit. This provided a much lower lead time, since code reviews were no longer being done before a merge - instead requiring our testing suite to pass. This fostered a culture, where each of us would be able to work on a task without any major repercussions, thereby increasing both risk-taking and experimentation. Another aspect is DevOps relates to our ability to automate repetitive tasks in an effort to reduce the time spent of a task, but in extension also avoid human errors, that might occur when multistep workflows needs are required.  